\chapter{4\quad Results and Discussion}
\label{ch:results_discussion}

\section{Comparison of Social Network Analysis across regions}
\label{sec:res_sna}

\subsection{One-way ANOVAs for network statistics}
\label{sec:sna_anovas}

One of the goals of this work is to investigate whether there are significant differences between the different datasets in terms of network statistics. This knowledge might be used e.g. to the impact of different containment methods. One-way ANOVAs with multiple comparisons were used to identify significant differences between the networks derived from the datasets. Furthermore, descriptive statistics are given for all computed network statistics.

\paragraph{Degree centrality} Recall that degree centrality values are normalised; therefore comparison values derived from original publications are adjusted accordingly. Average values for the different regions were $0.0071\pm0.016$ for the Yunnan dataset ($1.345\pm3.187$ reported in the original publication \cite{hainan_publication}; $0.0079\pm0.018$ adjusted for normalisation), $0.0093\pm0.011$ for the Hainan dataset ($1.51\pm1.79$ reported in the original publication \cite{hainan_publication}; $0.0093\pm0.011$ adjusted for normalisation), $0.0043\pm0.0056$ for the Shaanxi dataset ($0.987\pm1.351$ reported in the original publication \cite{shaanxi_publication}; $0.0041\pm0.0057$ adjusted for normalisation), $0.00036\pm0.00067$ for the Xi'an dataset ($0.741\pm1.379$ in the original publication \cite{xian_publication}; $0.00036\pm0.00067$ adjusted for normalisation), $1.80\cdot 10^{-5}\pm6.84\cdot 10^{-5}$ for the China dataset (no data reported in the original publication), and $7.9\cdot 10^{-6}\pm1.4\cdot 10^{-5}$ for the Bucharest dataset (no data reported in the original publication). Therefore, values could reliably be reproduced where applicable, assuming some loss of precision.

\begin{table}[h]
	\begin{mdframed}
		\begin{tabular*}{\linewidth}{l|llllll}
			\hline
			\textbf{ } & \textbf{Yunnan} & \textbf{Hainan} & \textbf{Shaanxi} & \textbf{Xi'an} & \textbf{Bucharest} & \textbf{China}\\
			\hline
			\textbf{N} & 171 & 162 & 237 & 2050 & 57836 & 25877\\
			\textbf{Minimum} & 0.0 & 0.0 & 0.0 & 0.0 & 0.0 & 0.0\\
			\textbf{25 Percentile} & 0.0 & 0.0 & 0.0 & 0.0 & 0.0 & 0.0\\
			\textbf{Median} & 0.0 & 0.0062 & 0.0042 & 0.00048 & 0.0 & 0.0\\
			\textbf{75 Percentile} & 0.0058 & 0.018 & 0.0084 & 0.00048 & $1.72\cdot 10^{-5}$ & 0.0\\
			\textbf{Maximum} & 0.070 & 0.037 & 0.046 & 0.020 & 0.00032 & 0.0052\\
			\textbf{Mean} & 0.0071 & 0.0093 & 0.0043 & 0.00036 & $7.93\cdot 10^{-6}$ & $1.80\cdot 10^{-5}$\\
			\textbf{Std. Dev.} & 0.016 & 0.011 & 0.0056 & 0.00067 & $1.42\cdot 10^{-5}$ & $6.74\cdot 10^{-5}$\\
			\hline
		\end{tabular*}
		\caption{Descriptive statistics for degree centrality.}
		\label{tab:degree_centrality_desc}
		\vskip 10pt
		\small
		\begin{tabular*}{\linewidth}{l|llll}
			\hline
			\textbf{Pair} & \textbf{Mean Diff.} & \textbf{95\% CI of diff.} & \textbf{Sign.?} & \textbf{Adj. P value}\\
			\hline
			Yunnan vs. Hainan & -0.0022 & -0.0024 to -0.0019 & Yes & $<0.0001$\\
			Yunnan vs. Shaanxi & 0.0028 & 0.0025 to 0.0030 & Yes & $<0.0001$\\
			Yunnan vs. Bucharest & 0.0071 & 0.0069 to 0.0073 & Yes & $<0.0001$\\
			Yunnan vs. China & 0.0071 & 0.0069 to 0.0073 & Yes & $<0.0001$\\
			Yunnan vs. Xi'an & 0.0067 & 0.0065 to 0.0070 & Yes & $<0.0001$\\
			Hainan vs. Shaanxi & 0.0050 & 0.0047 to 0.0053 & Yes & $<0.0001$\\
			Hainan vs. Bucharest & 0.0093 & 0.0091 to 0.0095 & Yes & $<0.0001$\\
			Hainan vs. China & 0.0093 & 0.0091 to 0.0095 & Yes & $<0.0001$\\
			Hainan vs. Xi'an & 0.0089 & 0.0087 to 0.0092 & Yes & $<0.0001$\\
			Shaanxi vs. Bucharest & 0.0043 & 0.0041 to 0.0044 & Yes & $<0.0001$\\
			Shaanxi vs. China & 0.0043 & 0.0041 to 0.0044 & Yes & $<0.0001$\\
			Shaanxi vs. Xi'an & 0.0039 & 0.0037 to 0.0041 & Yes & $<0.0001$\\
			Bucharest vs. China & $-1.01\cdot 10^{-5}$ & $-3.00\cdot 10^{-5}$ to $9.76\cdot 10^{-6}$ & No & 0.6956\\
			Bucharest vs. Xi'an & -0.00035 & -0.00041 to -0.00029 & Yes & $<0.0001$\\
			China vs. Xi'an & -0.00034 & -0.00040 to -0.00028 & Yes & $<0.0001$\\
			\hline
		\end{tabular*}
		\caption{Multiple comparisons for degree centrality. Method is Tukey's multiple comparison test with an alpha value of 0.05 (i.e. 95\% confidence interval).}
		\label{tab:degree_centrality_tukey}
	\end{mdframed}
\end{table}

One-way ANOVA comparison between the regions found a significant difference between the means ($F(5,86327) = 6265; \: p<0.0001$). Compared against each other (using Tukey's multiple comparisons test with 95\% confidence interval), all regions were pairwise significantly different except Bucharest vs. China ($p=0.6956$).

\paragraph{Betweenness centrality} Average values for the different regions were $7.32\cdot 10^{-6}\pm3.67\cdot 10^{-5}$ for the Yunnan dataset ($0.0000\pm0.0000$ reported in the original publication \cite{hainan_publication}), $1.72\cdot 10^{-5}\pm1.09\cdot 10^{-4}$ for the Hainan dataset ($0.0000\pm0.0001$ reported in the original publication \cite{hainan_publication}), $1.68\cdot 10^{-5}\pm1.02\cdot 10^{-4}$ for the Shaanxi dataset ($0.0000\pm0.0001$ reported in the original publication \cite{shaanxi_publication}), $1.60\cdot 10^{-6}\pm2.41\cdot 10^{-5}$ for the Xi'an dataset (no data reported in the original publication), $5.18\cdot 10^{-5}\pm0.01$ for the Bucharest dataset (no data reported in the original publication), and $2.33\cdot 10^{-8}\pm9.05\cdot 10^{-7}$ for the China dataset (no data reported in the original publication). Taking into account that values only include four digits after the decimal point and the resulting loss of information, values could reliably be reproduced where applicable.

\begin{table}[h]
	\begin{mdframed}
		\begin{tabular*}{\linewidth}{l|llllll}
			\hline
			\textbf{ } & \textbf{Yunnan} & \textbf{Hainan} & \textbf{Shaanxi} & \textbf{Xi'an} & \textbf{Bucharest} & \textbf{China}\\
			\hline
			\textbf{N} & 171 & 162 & 237 & 2050 & 57836 & 25877\\
			\textbf{Minimum} & 0.0 & 0.0 & 0.0 & 0.0 & 0.0 & 0.0\\
			\textbf{25 Percentile} & 0.0 & 0.0 & 0.0 & 0.0 & 0.0 & 0.0\\
			\textbf{Median} & 0.0 & 0.0 & 0.0 & 0.0 & 0.0 & 0.0\\
			\textbf{75 Percentile} & 0.0 & 0.0 & 0.0 & 0.0 & 0.0 & 0.0\\
			\textbf{Maximum} & 0.00034 & 0.0012 & 0.0013 & 0.00079 & 3.0 & $9.40\cdot 10^{-5}$\\
			\textbf{Mean} & $7.32\cdot 10^{-6}$ & $1.72\cdot 10^{-5}$ & $1.68\cdot 10^{-5}$ & $1.60\cdot 10^{-6}$ & $5.18\cdot 10^{-5}$ & $2.33\cdot 10^{-8}$\\
			\textbf{Std. Dev.} & $3.67\cdot 10^{-5}$ & 0.00010 & 0.00010 & $2.41\cdot 10^{-5}$ & 0.012 & $9.05\cdot 10^{-7}$\\
			\hline
		\end{tabular*}
		\caption{Descriptive statistics for betweenness centrality.}
		\label{tab:betweenness_centrality_desc}
		\vskip 10pt
		\small
		\begin{tabular*}{\linewidth}{l|llll}
			\hline
			\textbf{Pair} & \textbf{Mean Diff.} & \textbf{95\% CI of diff.} & \textbf{Sign.?} & \textbf{Adj. P value}\\
			\hline
			Yunnan vs. Hainan & $-9.92\cdot 10^{-6}$ & -0.0032 to 0.0031 & No & $>0.9999$\\
			Yunnan vs. Shaanxi & $-9.56\cdot 10^{-6}$ & -0.0029 to 0.0029 & No & $>0.9999$\\
			Yunnan vs. Bucharest & $-4.45\cdot 10^{-5}$ & -0.0022 to 0.0021 & No & $>0.9999$\\
			Yunnan vs. China & $7.30\cdot 10^{-6}$ & -0.0022 to 0.0022 & No & $>0.9999$\\
			Yunnan vs. Xi'an & $5.72\cdot 10^{-6}$ & -0.0023 to 0.0023 & No & $>0.9999$\\
			Hainan vs. Shaanxi & $3.63\cdot 10^{-7}$ & -0.0029 to 0.0029 & No & $>0.9999$\\
			Hainan vs. Bucharest & $-3.46\cdot 10^{-5}$ & -0.0023 to 0.0022 & No & $>0.9999$\\
			Hainan vs. China & $1.72\cdot 10^{-5}$ & -0.0022 to 0.0023 & No & $>0.9999$\\
			Hainan vs. Xi'an & $1.56\cdot 10^{-5}$ & 0.0023 to 0.0023 & No & $>0.9999$\\
			Shaanxi vs. Bucharest & $-3.49\cdot 10^{-5}$ & -0.0019 to 0.0018 & No & $>0.9999$\\
			Shaanxi vs. China & $1.68\cdot 10^{-5}$ & -0.0018 to 0.0019 & No & $>0.9999$\\
			Shaanxi vs. Xi'an & $1.52\cdot 10^{-5}$ & -0.0019 to 0.0020 & No & $>0.9999$\\
			Bucharest vs. China & $5.18\cdot 10^{-5}$ & -0.00016 to 0.00026 & No & 0.9843\\
			Bucharest vs. Xi'an & $5.02\cdot 10^{-5}$ & -0.00060 to 0.00070 & No & $>0.9999$\\
			China vs. Xi'an & $-1.58\cdot 10^{-6}$ & -0.00066 to 0.00066 & No & $>0.9999$\\
			\hline
		\end{tabular*}
		\caption{Multiple comparisons for betweenness centrality. Method is Tukey's multiple comparison test with an alpha value of 0.05 (i.e. 95\% confidence interval).}
		\label{tab:betweenness_centrality_tukey}
	\end{mdframed}
\end{table}

One-way ANOVA comparison between the regions found no significant difference between the means $F(5,86327) = 0.09719; \: p=0.9926$). Compared against each other (Tukey's test, 95\% CI), no regions were pairwise significantly different.

\paragraph{Pagerank centrality} Average values for the different regions were $0.0058\pm0.0055$ for the Yunnan dataset ($0.0058\pm0.0055$ reported in the original publication \cite{hainan_publication}), $0.0061\pm0.0043$ for the Hainan dataset ($0.0062\pm0.0043$ reported in the original publication \cite{hainan_publication}), $0.0042\pm0.0034$ for the Shaanxi dataset ($0.0042\pm0.0035$ reported in the original publication \cite{shaanxi_publication}), $0.00048\pm0.00057$ for the Xi'an dataset ($0.0005\pm0.0006$ reported in the original publication \cite{xian_publication}), $6.91\cdot 10^{-5}\pm0.01$ for the Bucharest dataset (no data reported in the original publication), and $3.86\cdot 10^{-5}\pm6.07\cdot 10^{-5}$ for the China dataset (no data reported in the original publication). Therefore, values could reliably be reproduced where applicable, assuming some loss of precision.

\begin{table}[h]
	\begin{mdframed}
		\begin{tabular*}{\linewidth}{l|llllll}
			\hline
			\textbf{ } & \textbf{Yunnan} & \textbf{Hainan} & \textbf{Shaanxi} & \textbf{Xi'an} & \textbf{Bucharest} & \textbf{China}\\
			\hline
			\textbf{N} & 171 & 162 & 237 & 2050 & 57836 & 25877\\
			\textbf{Minimum} & 0.0020 & 0.0014 & 0.00099 & 0.00012 & $6.09\cdot 10^{-6}$ & $1.87\cdot 10^{-5}$\\
			\textbf{25 Percentile} & 0.0020 & 0.0014 & 0.00099 & 0.00012 & $6.09\cdot 10^{-6}$ & $1.87\cdot 10^{-5}$\\
			\textbf{Median} & 0.0020 & 0.0081 & 0.0043 & 0.00044 & $6.09\cdot 10^{-6}$ & $1.87\cdot 10^{-5}$\\
			\textbf{75 Percentile} & 0.013 & 0.0098 & 0.0066 & 0.00081 & $2.60\cdot 10^{-5}$ & $1.87\cdot 10^{-5}$\\
			\textbf{Maximum} & 0.022 & 0.018 & 0.023 & 0.016 & 3.0 & 0.0047\\
			\textbf{Mean} & 0.0058 & 0.0061 & 0.0042 & 0.00048 & $6.91\cdot 10^{-5}$ & $3.86\cdot 10^{-5}$\\
			\textbf{Std. Dev.} & 0.0055 & 0.0043 & 0.0034 & 0.00057 & 0.012 & $6.07\cdot 10^{-5}$\\
			\hline
		\end{tabular*}
		\caption{Descriptive statistics for pagerank centrality.}
		\label{tab:pagerank_centrality_desc}
		\vskip 10pt
		\small
		\begin{tabular*}{\linewidth}{l|llll}
			\hline
			\textbf{Pair} & \textbf{Mean Diff.} & \textbf{95\% CI of diff.} & \textbf{Sign.?} & \textbf{Adj. P value}\\
			\hline
			Yunnan vs. Hainan & -0.00032 & -0.0035 to 0.0028 & No & 0.9997\\
			Yunnan vs. Shaanxi & 0.0016 & -0.0012 to 0.0045 & No & 0.6061\\
			Yunnan vs. Bucharest & 0.0057 & 0.0035 to 0.0080 & Yes & $<0.0001$\\
			Yunnan vs. China & 0.0058 & 0.0035 to 0.0080 & Yes & $<0.0001$\\
			Yunnan vs. Xi'an & 0.0053 & 0.0030 to 0.0076 & Yes & $<0.0001$\\
			Hainan vs. Shaanxi & 0.0019 & -0.0010 to 0.0049 & No & 0.4173\\
			Hainan vs. Bucharest & 0.0061 & 0.0038 to 0.0083 & Yes & $<0.0001$\\
			Hainan vs. China & 0.0061 & 0.0038 to 0.0084 & Yes & $<0.0001$\\
			Hainan vs. Xi'an & 0.0056 & 0.0033 to 0.0080 & Yes & $<0.0001$\\
			Shaanxi vs. Bucharest & 0.0041 & 0.0022 to 0.0060 & Yes & $<0.0001$\\
			Shaanxi vs. China & 0.0041 & 0.0022 to 0.0060 & Yes & $<0.0001$\\
			Shaanxi vs. Xi'an & 0.0037 &0.0017 to 0.0057 & Yes & $<0.0001$\\
			Bucharest vs. China & $3.05\cdot 10^{-5}$ & -0.00018 to 0.00024 & No & 0.9987\\
			Bucharest vs. Xi'an & -0.00041 & -0.0010 to 0.00023 & No & 0.4507\\
			China vs. Xi'an & -0.00044 & -0.0011 to 0.00021 & No & 0.3922\\
			\hline
		\end{tabular*}
		\caption{Multiple comparisons for pagerank centrality. Method is Tukey's multiple comparison test with an alpha value of 0.05 (i.e. 95\% confidence interval).}
		\label{tab:pagerank_centrality_tukey}
	\end{mdframed}
\end{table}

One-way ANOVA comparison between the regions found a significant difference between the means ($F(5,86327) = 38.86; \: p<0.0001$). Compared against each other (Tukey's test, 95\% CI), results were mixed. There was no significant difference for Yunnan vs. Hainan ($p=0.9997$), Yunnan vs. Shaanxi ($p=0.6061$), Hainan vs. Shaanxi ($p=0.4173$), Bucharest vs. China ($p=0.9987$), Bucharest vs. Xi'an ($p=0.4507$), and China vs. Xi'an ($p=0.3922$). All other pairs were significantly different.

\paragraph{Component size} Average values for the different regions were $2.42\pm3.23$ for the Yunnan dataset, $2.85\pm2.32$ for the Hainan dataset, $2.86\pm2.71$ for the Shaanxi dataset, $5.23\pm12.34$ for the Xi'an dataset, $1.93\pm1.79$ for the Bucharest dataset, and $6.76\pm38.15$ for the China dataset. No values were reported in the original publications, only individual clusters were examined in an exploratory fashion in \cite{hainan_publication,shaanxi_publication,xian_publication}.

\begin{table}[h]
	\begin{mdframed}
		\begin{tabular*}{\linewidth}{l|llllll}
			\hline
			\textbf{ } & \textbf{Yunnan} & \textbf{Hainan} & \textbf{Shaanxi} & \textbf{Xi'an} & \textbf{Bucharest} & \textbf{China}\\
			\hline
			\textbf{N} & 171 & 162 & 237 & 2050 & 57836 & 25877\\
			\textbf{Minimum} & 1.0 & 1.0 & 1.0 & 1.0 & 1.0 & 1.0\\
			\textbf{25 Percentile} & 1.0 & 1.0 & 1.0 & 1.0 & 1.0 & 1.0\\
			\textbf{Median} & 1.0 & 2.0 & 2.0 & 2.0 & 1.0 & 1.0\\
			\textbf{75 Percentile} & 2.0 & 4.0 & 4.0 & 4.0 & 2.0 & 1.0\\
			\textbf{Maximum} & 13.0 & 9.0 & 12.0 & 64.0 & 20.0 & 329.0\\
			\textbf{Mean} & 2.42 & 2.85 & 2.86 & 5.23 & 1.93 & 6.76\\
			\textbf{Std. Dev.} & 3.23 & 2.32 & 2.71 & 12.34 & 1.79 & 38.15\\
			\hline
		\end{tabular*}
		\caption{Descriptive statistics for component size.}
		\label{tab:component_size_desc}
		\vskip 10pt
		\small
		\begin{tabular*}{\linewidth}{l|llll}
			\hline
			\textbf{Pair} & \textbf{Mean Diff.} & \textbf{95\% CI of diff.} & \textbf{Sign.?} & \textbf{Adj. P value}\\
			\hline
			Yunnan vs. Hainan & -0.42 & -6.99 to 6.14 & No & $>0.9999$\\
			Yunnan vs. Shaanxi & -0.43 & -6.45 to 5.57 & No & $>0.9999$\\
			Yunnan vs. Bucharest & 0.49 & -4.09 to 5.08 & No & 0.9996\\
			Yunnan vs. China & -4.33 & -8.93 to 0.25 & No & 0.0772\\
			Yunnan vs. Xi'an & -2.81 & -7.58 to 1.95 & No & 0.5452\\
			Hainan vs. Shaanxi & -0.013 & -6.12 to 6.09 & No & $>0.9999$\\
			Hainan vs. Bucharest & 0.92 & -3.79 to 5.63 & No & 0.9936\\
			Hainan vs. China & -3.91 & -8.63 to 0.80 & No & 0.1698\\
			Hainan vs. Xi'an & -2.38 & -7.27 to 2.50 & No & 0.7328\\
			Shaanxi vs. Bucharest & 0.93 & -2.96 to 4.83 & No & 0.9838\\
			Shaanxi vs. China & -3.90 & -7.81 to 0.0087 & No & 0.0509\\
			Shaanxi vs. Xi'an & -2.37 & -6.48 to 1.73 & No & 0.5684\\
			Bucharest vs. China & -4.83 & -5.28 to -4.38 & Yes & $<0.0001$\\
			Bucharest vs. Xi'an & -3.30 & -4.65 to -1.96 & Yes & $<0.0001$\\
			China vs. Xi'an & 1.52 & 0.15 to 2.90 & Yes & 0.0192\\
			\hline
		\end{tabular*}
		\caption{Multiple comparisons for component sizes. Method is Tukey's multiple comparison test with an alpha value of 0.05 (i.e. 95\% confidence interval).}
		\label{tab:component_size_tukey}
	\end{mdframed}
\end{table}

One-way ANOVA comparison between the regions found a significant difference between the means ($F(5,86326) = 192.3; \: p<0.0001$). Compared against each other (Tukey's test, 95\% CI), no pairs were significantly different from each other except Bucharest vs. China, Bucharest vs. Xi'an, and China vs. Xi'an.

\paragraph{Average shortest path length} Average values for the different regions were $0.25\pm0.39$ for the Yunnan dataset, $0.45\pm0.47$ for the Hainan dataset, $0.52\pm0.56$ for the Shaanxi dataset, $0.61\pm0.80$ for the Xi'an dataset, $0.32\pm0.51$ for the Bucharest dataset, and $0.24\pm0.67$ for the China dataset. Shortest path length was not analysed in any of the previous works, therefore there are no comparison values.

\begin{table}[h]
	\begin{mdframed}
		\begin{tabular*}{\linewidth}{l|llllll}
			\hline
			\textbf{ } & \textbf{Yunnan} & \textbf{Hainan} & \textbf{Shaanxi} & \textbf{Xi'an} & \textbf{Bucharest} & \textbf{China}\\
			\hline
			\textbf{N} & 171 & 162 & 237 & 2050 & 57836 & 25877\\
			\textbf{Minimum} & 0.0 & 0.0 & 0.0 & 0.0 & 0.0 & 0.0\\
			\textbf{25 Percentile} & 0.0 & 0.0 & 0.0 & 0.0 & 0.0 & 0.0\\
			\textbf{Median} & 0.0 & 0.5 & 0.5 & 0.5 & 0.0 & 0.0\\
			\textbf{75 Percentile} & 0.5 & 0.75 & 1.0 & 1.0 & 0.5 & 0.0\\
			\textbf{Maximum} & 1.53 & 1.88 & 2.0 & 4.62 & 4.25 & 7.24\\
			\textbf{Mean} & 0.25 & 0.45 & 0.52 & 0.61 & 0.32 & 0.24\\
			\textbf{Std. Dev.} & 0.39 & 0.47 & 0.56 & 0.80 & 0.51 & 0.67\\
			\hline
		\end{tabular*}
		\caption{Descriptive statistics for average shortest path length per vertex.}
		\label{tab:avg_shortest_path_desc}
		\vskip 10pt
		\small
		\begin{tabular*}{\linewidth}{l|llll}
			\hline
			\textbf{Pair} & \textbf{Mean Diff.} & \textbf{95\% CI of diff.} & \textbf{Sign.?} & \textbf{Adj. P value}\\
			\hline
			Yunnan vs. Hainan & -0.20 & -0.38 to -0.023 & Yes & 0.0164\\
			Yunnan vs. Shaanxi & -0.26 & -0.43 to -0.10 & Yes & $<0.0001$\\
			Yunnan vs. Bucharest & -0.070 & -0.19 to 0.055 & No & 0.5994\\
			Yunnan vs. China & 0.0093 & -0.11 to 0.13 & No & $>0.9999$\\
			Yunnan vs. Xi'an & -0.35 & -0.48 to -0.22 & Yes & $<0.0001$\\
			Hainan vs. Shaanxi & -0.066 & -0.23 to 0.10 & No & 0.8659\\
			Hainan vs. Bucharest & 0.13 & 0.0034 to 0.26 & Yes & 0.0401\\
			Hainan vs. China & 0.21 & 0.083 to 0.34 & Yes & $<0.0001$\\
			Hainan vs. Xi'an & -0.15 & -0.29 to -0.022 & Yes & 0.0111\\
			Shaanxi vs. Bucharest & 0.19 & 0.092 to 0.30 & Yes & $<0.0001$\\
			Shaanxi vs. China & 0.27 & 0.17 to 0.38 & Yes & $<0.0001$\\
			Shaanxi vs. Xi'an & -0.089 & -0.20 to 0.022 & No & 0.2042\\
			Bucharest vs. China & 0.079 & 0.067 to 0.092 & Yes & $<0.0001$\\
			Bucharest vs. Xi'an & -0.28 & -0.32 to -0.25 & Yes & $<0.0001$\\
			China vs. Xi'an & -0.36 & -0.40 to -0.33 & Yes & $<0.0001$\\
			\hline
		\end{tabular*}
		\caption{Multiple comparisons for average shortest path length per vertex. Method is Tukey's multiple comparison test with an alpha value of 0.05 (i.e. 95\% confidence interval).}
		\label{tab:avg_shortest_path_tukey}
	\end{mdframed}
\end{table}

One-way ANOVA comparison between the regions found a significant difference between the means ($F(5,86326)=196.8; \: p<0.0001$). Compared against each other (Tukey's test, 95\% CI), results were mixed. There was no significant difference for Yunnan vs. Bucharest ($p=0.5994$), Yunnan vs. China ($p>0.9999$), Hainan vs. Shaanxi ($p=0.8659$), and Shaanxi vs Xi'an ($p=0.2042$). All other pairs were significantly different.

\subsection{Regression models for contact nomination likelihood}
\label{sec:sna_regression}

As briefly stated in chapter \ref{ch:methods}, ordinary least-squares regression was subsequently used to determine the influence of network statistics and actor covariates on contact nomination likelihood. To this end, for each dataset, all possible actor combinations were compiled into a dataset; each record contains network statistics and actor covariates of the source and target, and the absolute difference between source and target values for each variable. The regression model is fitted on the complete dataset, conditioned on whether there is a contact nomination seen in the actual underlying data (i.e. the original data presented in chapter \ref{ch:previous_work_data}). Therefore, the focus lies on explaining rather than predicting contact nominations. As covariates differ between the datasets, a different model was created for each dataset. The Xi'an dataset was excluded from this analysis, as it unfortunately does not include patient covariates.

\paragraph{Yunnan network} With a 95\% confidence interval, the statistically significant predictors for contact nomination likelihood are the sex of the nominator ($p<0.001$), sex discrepancy between nominator and nominee ($p<0.001$), whether the nominator is a family member of another actor in the dataset ($p<0.001$), and all network statistics for nominator, nominee, and the difference between nominator and nominee, with the exception of the average shortest path length of the nominator ($p=0.224$). Nominator sex has a marginally positive effect on contact nomination likelihood (coefficient = 0.0036); sex discrepancy has a marginally positive effect (coefficient = 0.0033); family member status of the nominator has a marginally positive effect (coefficient = 0.0036); degree centrality of nominator and nominee both have a slightly positive effect (coefficients = 0.0874 and 0.0901), while degree centrality discrepancy has a slightly negative effect (coefficient = -0.0880); betweenness centrality of nominator and nominee both have a slightly negative effect (covariates = -0.0103 and -0.0114), while betweenness centrality discrepancy has a slightly positive effect (coefficient = 0.0182); pagerank centrality of the nominator has a slightly negative effect, the pagerank centrality of the nominee has a slightly positive effect, and pagerank centrality discrepancy has a slightly negative effect (coefficients = -0.0182, 0.172, and -0.0193, respectively); the component size of the nominator has a slightly negative effect, the component size of the nominee has a slightly positive effect, and component size discrepancy has a slightly negative effect (coefficients = -0.0350, 0.1101, and -0.1140, respectively); finally, the average shortest path length of the nominator has a marginally positive effect, the average shortest path length of the nominee has a slightly negative effect, and average shortest path length discrepancy has a slightly positive effect on contact nomination likelihood (coefficients = 0.0024, -0.0665, and 0.0702, respectively).

\begin{table}[h]
	\centering
	\begin{mdframed}
		\begin{tabular}[width=\linewidth]{l|llll}
			\hline
			& \bfseries coef & \bfseries std err & $\mathbf{t}$ & $\mathbf{P>\lvert t \rvert}$\\
			\hline
			\csvreader[head to column names]{Tables/yunnan_regression.csv}{}
			{\\ \a & \b & \c & \d & \e}\\
			\hline
		\end{tabular}
		\caption{Ordinary least-squares regression results for the Yunnan network.}
		\label{tab:yunnan_regression}
	\end{mdframed}

\end{table}

\paragraph{Hainan network} With a 95\% confidence interval, the statistically significant predictors for contact nomination likelihood are whether the nominee is a family member of another actor in the datatset ($p<0.001$) and all network statistics with the exceptions of nominee degree centrality ($p=0.435$), nominee pagerank centrality ($p=0.705$), pagerank centrality discrepancy ($p=0.126$), and nominator component size ($p=0.790$). Nominee family member status has a marginally positive effect on contact nomination likelihood (coefficient = 0.0132); nominator degree centrality has a marginally positive effect, while degree centrality discrepancy has a marginally negative effect (coefficients = 0.0385 and -0.0125, respectively); nominator and nominee betweenness centrality have a marginally positive effect, while betweenness centrality discrepancy has a marginally negative effect (coefficients = 0.0431, 0.0733, and =0.0820, respectively); nominator pagerank centrality has a marginally negative effect (coefficient = -0.0114); the component size of the nominee has a marginally positive effect, while component size discrepancy has a marginally negative effect (coefficients = 0.0506 and -0.0487, respectively); nominator and nominee average path length both have a marginally negative effect, while average path length discrepancy has a marginally positive effect (coefficients = -0.0105, -0.0268 and 0.0289, respectively).

\begin{table}[h]
	\centering
	\begin{mdframed}
		\begin{tabular}[width=\linewidth]{l|llll}
			\hline
			& \bfseries coef & \bfseries std err & $\mathbf{t}$ & $\mathbf{P>\lvert t \rvert}$\\
			\hline
			\csvreader[head to column names]{Tables/hainan_regression.csv}{}
			{\\ \a & \b & \c & \d & \e}\\
			\hline
		\end{tabular}
		\caption{Ordinary least-squares regression results for the Hainan network.}
		\label{tab:hainan_regression}
	\end{mdframed}
\end{table}

\paragraph{Shaanxi network} With a 95\% confidence interval, the statistically significant predictors for contact nomination likelihood are the age of the nominator ($p=0.016$), the age of the nominee ($p=0.008$), age discrepancy ($p=0.007$), sex of the nominator ($p=0.001$), sex of the nominee ($p<0.001$), sex discrepancy ($p<0.001$), whether the nominator is a family member of another actor in the dataset ($p<0.001$), family member status discrepancy ($p=0.045$), the place of residence of the nominator ($p<0.001$), the place of residence of the nominee ($p<0.001$), place of residence discrepancy ($p<0.001$), and all network statistics except pagerank centrality discrepancy ($p=0.244$). Nominator age, nominee age, and age discrepancy all have a marginally negative effect on contact nomination likelihood (coefficients = -0.0033, -0.0032 and -0.0028, respectively); nominator sex, nominee sex and sex discrepancy all have a marginally positive effect (coefficients = 0.0025, 0.0028 and 0.0040, respectively); the family member status of the nominator and family member discrepancy both have a marginally positive effect (coefficients = 0.0107 and 0.0017, respectively); nominator and nominee place of residence have a marginally positive effect, while place of residence discrepancy has a marginally negative effect (coefficients = 0.0007, 0.0006 and -0.0145, respectively); nominator and nominee degree centrality have a marginally positive effect, while degree centrality discrepancy has a marginally negative effect (coefficients = 0.0171, 0.0101 and -0.0110, respectively); nominator betweenness centrality has a marginally negative effect, while nominee betweenness centrality and betweenness centrality discrepancy have a marginally positive effect (coefficients = -0.0163, 0.0034 and 0.0204, respectively); nominator and nominee pagerank centrality both have a marginally negative effect (coefficients = -0.0103 and -0.0019, respectively); nominator and nominee component size have a marginally positive effect, while component size discrepancy has a marginally negative effect (coefficients = 0.0293, 0.0218 and -0.0381, respectively); nominator and nominee average path length have a marginally negative effect, while average path length discrepancy has a marginally positive effect (coefficients = -0.0162, -0.0166 and 0.0206, respectively).

\begin{table}[h]
	\centering
	\begin{mdframed}
		\begin{tabular}[width=\linewidth]{l|llll}
			\hline
			& \bfseries coef & \bfseries std err & $\mathbf{t}$ & $\mathbf{P>\lvert t \rvert}$\\
			\hline
			\csvreader[head to column names]{Tables/shanxi_regression.csv}{}
			{\\ \a & \b & \c & \d & \e}\\
			\hline
		\end{tabular}
		\caption{Ordinary least-squares regression results for the Shaanxi network.}
		\label{tab:shaanxi_regression}
	\end{mdframed}
\end{table}

\paragraph{Bucharest network} With a 95\% confidence interval, the statistically significant predictors for contact nomination likelihood are age discrepancy ($p<0.001$), sex of the nominator ($p<0.001$), sex of the nominee ($p<0.001$), sex discrepancy ($p<0.001$), whether the nominator is active in the medical field ($p<0.001$), whether the nominee is active in the medical field ($p<0.001$), medical employment discrepancy ($p<0.001$), the job of the nominator ($p<0.001$), the job of the nominee ($p<0.001$), job discrepancy ($p<0.001$), and all network statistics. Age discrepancy has a marginally negative effect on contact nomination likelihood (coefficient = -0.0290); nominator and nominee sex have a marginally positive effect, while sex discrepancy has a marginally negative effect (coefficients = 0.0008, 0.0002 and -0.0007, respectively); nominator and nominee medical deployment have a marginally/slightly negative effect, while medical employment discrepancy has a marginally positive effect (coefficients = -0.0827, -0.1616 and 0.0588, respectively); the job of the nominator has a marginally negative effect, while the job of the nominee and job discrepancy have a marginally/slightly positive effect (coefficients = -0.0084, 0.0086 and 0.1852, respectively); nominator degree centrality, nominee degree centrality and degree centrality discrepancy all have a moderately/slightly positive effect (coefficients = 0.3564, 0.1728 and 0.1853, respectively); nominator betweenness centrality has a marginally negative effect, while nominee betweenness centrality and betweenness centrality discrepancy have a marginally/slightly positive effect (coefficients = -0.0820, 0.0301 and 0.1155, respectively); nominator pagerank centrality has a marginally positive effect, while nominee pagerank centrality and pagerank centrality discrepancy have a slightly/moderate negative effect (coefficients = 0.0816, -0.1084 and -0.4072, respectively); nominator component size, nominee component size and component size discrepancy all have a slightly/marginally negative effect (coefficients = -0.2374, -0.0984 and -0.0769, respectively); finally, nominator and nominee average shortest path length both have a moderate/marginally positive effect, while average shortest path length discrepancy has a marginally negative effect (coefficients = 0.2308, 0.0424 and -0.0548, respectively).

\begin{table}[h]
	\centering
	\begin{mdframed}
		\begin{tabular}[width=\linewidth]{l|llll}
			\hline
			& \bfseries coef & \bfseries std err & $\mathbf{t}$ & $\mathbf{P>\lvert t \rvert}$\\
			\hline
			\csvreader[head to column names]{Tables/bucharest_regression.csv}{}
			{\\ \a & \b & \c & \d & \e}\\
			\hline
		\end{tabular}
		\caption{Ordinary least-squares regression results for the Bucharest network.}
		\label{tab:bucharest_regression}
	\end{mdframed}
\end{table}

\paragraph{China network} With a 95\% confidence interval, the statistically significant predictors for contact nomination likelihood are the age of the nominator ($p=0.032$), the age of the nominee ($p<0.001$), age discrepancy ($p<0.001$), the sex of the nominator ($p<0.001$), the sex of the nominee ($p<0.001$), sex discrepancy ($p<0.001$), the place of residence of the nominator ($p<0.001$), the place of residence of the nominee ($p<0.001$), place of residence discrepancy ($p<0.001$), the place/occasion where the nominator might have been infected ($p<0.001$), the place/occasion where the nominee might have been infected ($p<0.001$), place/occasion discrepancy ($p<0.001$), the nominator's symptoms ($p<0.001$), the nominee's symptoms ($p<0.001$), symptom discrepancy ($p<0.001$), the severity of the nominator's symptoms ($p<0.001$), the severity of the nominee's symptoms ($p<0.001$), symptom severity discrepancy ($p<0.001$), the discrepancy between the place where nominator and nominees were registered as patients ($p<0.001$), and all network statistics. The age of the nominator has a marginally negative effect on contact nomination likelihood, while nominee age and age discrepancy have a marginally positive effect (coefficients = -0.0046, 0.0130 and 0.0087, respectively); The sex of the nominator, the sex of the nominee and sex discrepancy all have a marginally positive effect (coefficients = 0.0144, 0.0161 and 0.0409, respectively); the nominator's place of residence, the nominee's place of residence and place of residence discrepancy all have a negligibly/marginally positive effect (coefficients = 0.0001, 0.0002 and 0.0136, respectively); the place/occasion where the nominator might have been infected and the place/occasion where the nominee might have been infected have a negligibly positive effect, while the discrepancy between the two has a marginally negative effect (coefficients = 0.000039, 0.000063 and -0.0148, respectively); the nominator's symptoms and the nominee's symptoms both have a negligibly positive effect, while the symptom discrepancy has a marginally negative effect (coefficients = 0.0005, 0.0006 and -0.0405, respectively); the severity of the nominator's symptoms, the severity of the nominee's symptoms and the discrepancy in symptom severity all have a marginally positive effect (coefficients = 0.0689, 0.0788 and 0.0999, respectively); the discrepancy between the places where nominator and nominee have been recorded as positive patients has a strong negative effect (coefficient = -0.7422); the degree centrality of the nominator has a marginally positive effect, while the degree centrality of the nominee and the degree centrality discrepancy have a marginally negative effect (coefficients = 0.0155, -0.0284 and -0.0368, respectively); the betweenness centrality of the nominator and the betweenness centrality of the nominee both have a negligibly/marginally positive effect, while the betweenness centrality discrepancy has a marginally negative effect (coefficients = 0.0037, 0.0841 and -0.0417, respectively); the pagerank centrality of the nominator and the pagerank centrality of the nominee have a marginally positive effect, while the pagerank centrality discrepancy has a marginally negative effect (coefficients = 0.0246, 0.1119 and -0.0497, respectively); the component size of the nominator and the component size of the nominee both have a marginally positive effect, while the component size discrepancy has a marginally negative effect (coefficients = 0.0060, 0.0295 and -0.0156, respectively); finally, the average shortest path length of the nominator and the average shortest path length of the nominee both have a marginally positive effect, while the average shortest path length discrepancy has a marginally negative effect (coefficients = 0.0420, 0.0236 and -0.0370, respectively);

\begin{table}[h]
	\centering
	\begin{mdframed}
		\begin{tabular}[width=\linewidth]{l|llll}
			\hline
			& \bfseries coef & \bfseries std err & $\mathbf{t}$ & $\mathbf{P>\lvert t \rvert}$\\
			\hline
			\csvreader[head to column names]{Tables/china_regression.csv}{}
			{\\ \a & \b & \c & \d & \e}\\
			\hline
		\end{tabular}
		\caption{Ordinary least-squares regression results for the China network.}
		\label{tab:china_regression}
	\end{mdframed}
\end{table}

\section{Comparison of Relational Event Model Analysis across regions}
\label{sec:res_rem}

All five main datasets (Yunnan, Hainan, Shaanxi, Bucharest and China) were analysed using relational event modelling. A varying number of non-events were sampled for each observed event, such that for each individual dataset, the total number of events would be around one million. Cox proportional hazards models were used to determine the significance and effect of computed statistics (which are listed in chapter \ref{ch:methods}) on contact nomination likelihood. Some of the statistics were all-zero in some datasets. The results of the Cox regressions will be presented below. For all datasets, three different sets of statistics and accompanying Cox proportional hazards models were computed. One with no conditioning, i.e. for any event $e(u,v,t_i)$, sampled non-events may include any other event $(u',v',t_i)$ which may have the same or a different source and the same or a different target, but not $u' = u \land v' = v$; one with source-specific conditioning, i.e. for any event $e(u,v,t_i)$, sampled non-events may only include those $e_s(u,v',t_i)$ which have the same source $u$, but another target $v'$; and one with target-specific conditioning, i.e. for any event $e(u,v,t_i)$, sampled non-events may only include those $e_s(u',v,t_i)$ which have the same target $v$, but another source $u'$. 

\paragraph{Yunnan network} With a 90\% confidence interval, the statistics that have a statistically significant impact on contact nomination likelihood are reciprocation ($p<0.0001$), the discrepancy between nominator's and nominee's age ($p<0.0001$), the sex of the nominee ($p=0.0302$), whether the nominee is a family member of a family member who was registered as a patient at a previous point in time ($p=0.0508$), and the discrepancy in family member status ($p<0.0001$). Reciprocation has a slightly positive effect on contact nomination likelihood (coefficient = 0.2434); age discrepancy has a marginally negative effect (coefficient = -0.0513), the sex of the nominee has a moderately positive effect (coefficient = 0.5077), the family member status of the nominee has a strong negative effect (coefficient = -1.127), and the discrepancy in family member status has a very strong negative effect (coefficient = -3.009).

\begin{table}[h]
	\centering
	\begin{mdframed}
		\begin{tabular}[width=\linewidth]{l|llll}
			\hline
			& \bfseries coef & \bfseries std err & $\mathbf{z}$ & $\mathbf{P>\lvert z \rvert}$\\
			\hline
			\csvreader[head to column names]{Tables/yunnan_rem.csv}{}
			{\\ \csvcoliii & \csvcoliv & \csvcolv & \csvcolvi & \csvcolvii}\\
			\hline
		\end{tabular}
		\caption{Unconditioned REM results for the Yunnan network.}
		\label{tab:yunnan_rem}
	\end{mdframed}
\end{table}

For the source-conditioned model, the statistics that have a statistically significant impact on contact nomination likelihood are reciprocation ($p<0.0001$), the age of the nominator ($p=0.0920$), the discrepancy between nominator's and nominee's age ($p<0.0001$), the sex of the nominee ($p=0.0082)$, whether the nominee is a family member of an actor who was registered as a patient at a previous point in time ($p=0.0438$), and the discrepancy in family member status $p<0.0001$. Reciprocation has a slightly positive effect on contact nomination likelihood (coefficient = 0.2936), the nominator's age has a marginally negative effect (coefficient = -0.0152), age discrepancy has a marginally negative effect (coefficient = -0.0497), the sex of the nominee has a moderately positive effect (coefficient = 0.6192), discrepancy in family member status has a strong negative effect (coefficient = -1.170), and discrepancy in family member status has a very
strong negative effect (coefficient = -3.094). Nomination activity, exact repetition, exact reciprocation, transitive tie, cyclical tie, shared sender and shared receiver are all-zero and thus weren't included in the Cox model.

\begin{table}[h]
	\centering
	\begin{mdframed}
		\begin{tabular}[width=\linewidth]{l|llll}
			\hline
			& \bfseries coef & \bfseries std err & $\mathbf{z}$ & $\mathbf{P>\lvert z \rvert}$\\
			\hline
			\csvreader[head to column names]{Tables/yunnan_rem_cond_sender.csv}{}
			{\\ \csvcoliii & \csvcoliv & \csvcolv & \csvcolvi & \csvcolvii}\\
			\hline
		\end{tabular}
		\caption{Source-conditioned REM results for the Yunnan network. Nomination activity, exact repetition, exact reciprocation, transitive tie, cyclical tie, shared sender and shared receiver are all-zero and thus weren't included in the Cox model.}
		\label{tab:yunnan_rem_cond_sender}
	\end{mdframed}
\end{table}

Finally, for the target-conditioned model, the statistics that have a statistically significant impact on contact nomination likelihood are the discrepancy between nominator's and nominee's age ($p=0.0001$) and the discrepancy in family member status ($p<0.0001$). Age discrepancy has a marginally negative effect on contact nomination likelihood (coefficient = -0.0412), and family member status discrepancy has a strong negative effect (coefficient = -2.898).

\begin{table}[h]
	\centering
	\begin{mdframed}
		\begin{tabular}[width=\linewidth]{l|llll}
			\hline
			& \bfseries coef & \bfseries std err & $\mathbf{z}$ & $\mathbf{P>\lvert z \rvert}$\\
			\hline
			\csvreader[head to column names]{Tables/yunnan_rem_cond_receiver.csv}{}
			{\\ \csvcoliii & \csvcoliv & \csvcolv & \csvcolvi & \csvcolvii}\\
			\hline
		\end{tabular}
		\caption{Target-conditioned REM results for the Yunnan network.}
		\label{tab:yunnan_rem_cond_receiver}
	\end{mdframed}
\end{table}

\paragraph{Hainan network} With a 90\% confidence interval, the statistics that have a statistically significant impact on contact nomination likelihood are the age of the nominator ($p=0.0152$), the age of the nominee ($p=0.0509$), the discrepancy between nominator's and nominee's age ($p=0.0256$), the sex of the nominee ($p=0.0762$), whether the nominator is a family member of an actor who was registered as a patient at a previous point in time ($p=0.0001$), whether the nominee is a family member of an actor who was registered as a patient at a previous point in time ($p<0.0001$), and the discrepancy in family member status ($p<0.0001$). The nominator's age, the nominee's age and age discrepancy all have a marginally negative effect on contact nomination likelihood (coefficients = -0.0121, -0.0095 and -0.0137, respectively); the sex of the nominee has a marginally positive effect (coefficient = 0.3250); family member status of the nominator and nominee both have a strongly positive effect, while family member status discrepancy has a strongly negative effect (coefficients = 1.504, 2.213 and -1.586, respectively).

\begin{table}[h]
	\centering
	\begin{mdframed}
		\begin{tabular}[width=\linewidth]{l|llll}
			\hline
			& \bfseries coef & \bfseries std err & $\mathbf{z}$ & $\mathbf{P>\lvert z \rvert}$\\
			\hline
			\csvreader[head to column names]{Tables/hainan_rem.csv}{}
			{\\ \csvcoliii & \csvcoliv & \csvcolv & \csvcolvi & \csvcolvii}\\
			\hline
		\end{tabular}
		\caption{Unconditioned REM results for the Hainan network.}
		\label{tab:hainan_rem}
	\end{mdframed}
\end{table}

For the source-conditioned model, the statistics that have a statistically significant impact on contact nomination likelihood are the age of the nominator ($p=0.0404$), the discrepancy between nominator's and nominee's age ($p=0.0132$), whether the nominator is a family member of another actor who was registered as a patient at a previous point in time ($p=0.0068$), and the discrepancy in family member status ($p<0.0001$). The nominator's age and age discrepancy both have a marginally negative effect on contact nomination likelihood (coefficients = -0.0102 and -0.0151, respectively); the family member status of the nominator has a strongly positive effect, while discrepancy in family member status has a strongly negative effect (coefficients = 1.065 and -1.580, respectively). Nomination activity, exact repetition, exact reciprocation, transitive tie, cyclical tie, shared sender and shared receiver are all-zero and thus weren't included in the Cox model.

\begin{table}[h]
	\centering
	\begin{mdframed}
		\begin{tabular}[width=\linewidth]{l|llll}
			\hline
			& \bfseries coef & \bfseries std err & $\mathbf{z}$ & $\mathbf{P>\lvert z \rvert}$\\
			\hline
			\csvreader[head to column names]{Tables/hainan_rem_cond_sender.csv}{}
			{\\ \csvcoliii & \csvcoliv & \csvcolv & \csvcolvi & \csvcolvii}\\
			\hline
		\end{tabular}
		\caption{Source-conditioned REM results for the Hainan network. Nomination activity, exact repetition, exact reciprocation, transitive tie, cyclical tie, shared sender and shared receiver are all-zero and thus weren't included in the Cox model.}
		\label{tab:hainan_rem_cond_sender}
	\end{mdframed}
\end{table}

Finally, for the target-conditioned model, the statistics that have a statistically significant impact on contact nomination likelihood are the age of the nominator ($p=0.0119$), the age of the nominee ($p=0.0996$), the discrepancy between nominator's and nominee's age ($p=0.0132$), whether the nominator is a family member of another actor who was registered as a patient at a previous point in time ($p=0.0002$), and the discrepancy in family member status ($p<0.0001$). The nominator's age, the nominee's age and age discrepancy all have a marginally negative effect on contact nomination likelihood (coefficients = -0.0125, -0.0091 and -0.0150, respectively); the family member status of the nominator has a strongly positive effect, while discrepancy in family member status has a strongly negative effect (coefficients = 1.449 and -1.664, respectively).

\begin{table}[h]
	\centering
	\begin{mdframed}
		\begin{tabular}[width=\linewidth]{l|llll}
			\hline
			& \bfseries coef & \bfseries std err & $\mathbf{z}$ & $\mathbf{P>\lvert z \rvert}$\\
			\hline
			\csvreader[head to column names]{Tables/hainan_rem_cond_receiver.csv}{}
			{\\ \csvcoliii & \csvcoliv & \csvcolv & \csvcolvi & \csvcolvii}\\
			\hline
		\end{tabular}
		\caption{Target-conditioned REM results for the Hainan network.}
		\label{tab:hainan_rem_cond_receiver}
	\end{mdframed}
\end{table}

\paragraph{Shaanxi network} With a 90\% confidence interval, the statistics that have a statistically significant impact on contact nomination likelihood are reciprocation ($p<0.0001$), the discrepancy between nominator's and nominee's sex ($p=0.0718$), whether the nominator is a family member of another actor who has been registered as a patient at a previous point in time ($p<0.0001$), whether the nominee is a family member of another actor who has been registered as a patient at a previous point in time ($p<0.0001$), the place of residence of the nominee ($p=0.0011$), and the discrepancy between nominator's and nominee's place of residence ($p<0.0001$). Reciprocation has a marginally positive effect on contact nomination likelihood (coefficient = 0.3591); sex discrepancy has a marginally positive effect (coefficient = 0.3333); family member status of nominator and nominee both have a strongly/marginally positive effect (coefficients = 2.377 and 0.0692, respectively); the nominee's place of residence and discrepancy in place of residence have a marginally/strongly negative effect (coefficients = -0.0624 and -3.086, respectively).

\begin{table}[h]
	\centering
	\begin{mdframed}
		\begin{tabular}[width=\linewidth]{l|llll}
			\hline
			& \bfseries coef & \bfseries std err & $\mathbf{z}$ & $\mathbf{P>\lvert z \rvert}$\\
			\hline
			\csvreader[head to column names]{Tables/shanxi_rem.csv}{}
			{\\ \csvcoliii & \csvcoliv & \csvcolv & \csvcolvi & \csvcolvii}\\
			\hline
		\end{tabular}
		\caption{Unconditioned REM results for the Shaanxi network.}
		\label{tab:shaanxi_rem}
	\end{mdframed}
\end{table}

For the source-conditioned model, the statistics that have a statistically significant impact on contact nomination likelihood are reciprocation ($p<0.0001$), discrepancy between nominator's and nominee's sex ($p=0.0815$), whether the nominee is a family member of another actor who has been registered as a patient at a previous point in time ($p<0.0001$), the place of residence of the nominee ($p=0.0007$), and the discrepancy between nominator's and nominee's place of residence ($p<0.0001$). Reciprocation has a moderately positive effect on contact nomination likelihood (coefficient = 0.3555); sex discrepancy has a moderately positive effect (coefficient = 0.3222); family member status of the nominee has a strongly positive effect (coefficient = 0.8221); the place of residence of the nominee and discrepancy in place of residence have a marginally/strongly negative effect (coefficients = -0.0654 and -3.071, respectively). Nomination activity, exact repetition, exact reciprocation, transitive tie, cyclical tie, shared sender and shared receiver are all-zero and thus weren't included in the Cox model.

\begin{table}[h]
	\centering
	\begin{mdframed}
		\begin{tabular}[width=\linewidth]{l|llll}
			\hline
			& \bfseries coef & \bfseries std err & $\mathbf{z}$ & $\mathbf{P>\lvert z \rvert}$\\
			\hline
			\csvreader[head to column names]{Tables/hainan_rem_cond_sender.csv}{}
			{\\ \csvcoliii & \csvcoliv & \csvcolv & \csvcolvi & \csvcolvii}\\
			\hline
		\end{tabular}
		\caption{Source-conditioned REM results for the Shaanxi network. Nomination activity, exact repetition, exact reciprocation, transitive tie, cyclical tie, shared sender and shared receiver are all-zero and thus weren't included in the Cox model.}
		\label{tab:shaanxi_rem_cond_sender}
	\end{mdframed}
\end{table}

Finally, for the target-conditioned model, the statistics that have a statistically significant impact on contact nomination likelihood are the discrepancy between nominator's and nominee's sex ($p-0.0935$), whether the nominator is a family member of another actor who has been registered as a patient at a previous point in time ($p<0.0001$), the place of residence of the nominee ($p=0.0104$), and the discrepancy between nominator's and nominee's place of residence ($p<0.0001$). Sex discrepancy has a marginally positive effect on contact nomination likelihood (coefficient = 0.3118); family member status of the nominator has a strongly positive effect (coefficient = 2.276); the place of residence of the nominee has a marginally negative effect (coefficient = -0.0638); the discrepancy in place of residence has a strongly negative effect (coefficient = -3.297).

\begin{table}[h]
	\centering
	\begin{mdframed}
		\begin{tabular}[width=\linewidth]{l|llll}
			\hline
			& \bfseries coef & \bfseries std err & $\mathbf{z}$ & $\mathbf{P>\lvert z \rvert}$\\
			\hline
			\csvreader[head to column names]{Tables/shanxi_rem_cond_receiver.csv}{}
			{\\ \csvcoliii & \csvcoliv & \csvcolv & \csvcolvi & \csvcolvii}\\
			\hline
		\end{tabular}
		\caption{Target-conditioned REM results for the Shaanxi network.}
		\label{tab:shaanxi_rem_cond_receiver}
	\end{mdframed}
\end{table}

\paragraph{Bucharest network} With a 90\% confidence interval, the statistics that have a statistically significant impact on contact nomination likelihood are reciprocation ($p<0.0001$), exact reciprocation ($p<0.0001$), cyclical tie ($p=0.0018$), shared sender ($p<0.0001$), the age of the nominator ($p=0.0305$), the age of the nominee ($p<0.0001$), the discrepancy between nominator's and nominee's age ($p<0.0001$), the sex of the nominator ($p=0.0051$), the sex of the nominee ($p<0.0001$), the discrepancy between nominator's and nominee's sex ($p<0.0001$), the nominee's job ($p<0.0001$), the discrepancy between nominator's and nominee's job ($p<0.0001$), whether the nominator is active in the medical field ($p<0.0001$), whether the nominee is active in the medical field ($p<0.0001$), and the discrepancy between nominator's and nominee's medical employment status ($p<0.0001$). Reciprocation has a slightly negative effect on contact nomination likelihood (coefficient = -0.3098); exact reciprocation has a marginally positive effect (coefficient = 0.0256); cyclical tie has a marginally positive effect (coefficient = 0.0027); shared sender has a marginally positive effect (coefficient = 0.0189); nominator age, nominee age and age discrepancy all have a marginally negative effect (coefficients = -0.0028, -0.0094 and -0.0220, respectively); nominator sex, nominee sex and sex discrepancy all have a slightly/moderately positive effect (coefficients = 0.1444, 0.2500 and 0.6859, respectively); nominee job has a marginally positive effect, while job discrepancy has a moderately negative effect (coefficients = 0.0577 and -0.6991, respectively); nominator and nominee medical employment status both have a strongly negative effect, while discrepancy in medical employment has a moderately positive effect (coefficients = -1.5437, -1.4391 and 0.4569, respectively). Transitive tie and shared target are all-zero and thus weren't included in the Cox model.

\begin{table}[h]
	\centering
	\begin{mdframed}
		\begin{tabular}[width=\linewidth]{l|llll}
			\hline
			& \bfseries coef & \bfseries std err & $\mathbf{z}$ & $\mathbf{P>\lvert z \rvert}$\\
			\hline
			\csvreader[head to column names]{Tables/bucharest_rem.csv}{}
			{\\ \csvcoliii & \csvcoliv & \csvcolv & \csvcolvi & \csvcolvii}\\
			\hline
		\end{tabular}
		\caption{Unconditioned REM results for the Bucharest network. Transitive tie and shared target are all/zero and thus weren't included in the Cox model.}
		\label{tab:bucharest_rem}
	\end{mdframed}
\end{table}

For the source-conditioned model, the statistics that have a statistically significant impact on contact nomination likelihood are reciprocation ($p<0.0001$), exact reciprocation ($p<0.0001$), cyclical tie ($p=0.0006$), shared source ($p<0.0001$), the age of the nominee ($p<0.0001$), the discrepancy between nominator's and nominee's age ($p<0.0001$), the sex of the nominator ($p=0.0004$), the sex of the nominee ($p<0.0001$), the discrepancy between nominator's and nominee's sex ($p<0.0001$), the job of the nominee ($p<0.0001$), the discrepancy between nominator's and nominee's job ($p<0.0001$), whether the nominator is employed in the medical field ($p<0.0001$), whether the nominee is employed in the medical field($p<0.0001$), and the discrepancy between nominator's and nominee's medical employment status ($p=0.0003$). Reciprocation has a slightly negative effect on contact nomination likelihood (coefficient = -0.3209), exact reciprocation has a marginally positive effect (coefficient = 0.0347), cyclical tie has a marginally positive effect (coefficient = 0.0030), shared source has a marginally positive effect (coefficient = 0.0185), nominee age and age discrepancy both have a marginally negative effect (coefficients = -0.0090 and -0.0118, respectively); nominator sex, nominee sex and sex discrepancy all have a slightly/moderately positive effect (coefficients = 0.1825, 0.2266 and 0.6970, respectively); nominee job has a marginally positive effect, while discrepancy between nominator's and nominee's job has a strongly negative effect (coefficients = 0.0656 and -0.7420, respectively); medical employment of nominator and nominee both have a slightly/strongly negative effect, while discrepancy in medical employment has a slightly positive effect (coefficients = -0.3595, -1.5172 and 0.3145, respectively). Nomination activity, repetition, transitive tie and shared target are all-zero and thus weren't included in the Cox model.

\begin{table}[h]
	\centering
	\begin{mdframed}
		\begin{tabular}[width=\linewidth]{l|llll}
			\hline
			& \bfseries coef & \bfseries std err & $\mathbf{z}$ & $\mathbf{P>\lvert z \rvert}$\\
			\hline
			\csvreader[head to column names]{Tables/bucharest_rem_cond_sender.csv}{}
			{\\ \csvcoliii & \csvcoliv & \csvcolv & \csvcolvi & \csvcolvii}\\
			\hline
		\end{tabular}
		\caption{Source-conditioned REM results for the Bucharest network. Nomination activity, repetition, transitive tie and shared target are all-zero and thus weren't included in the Cox model.}
		\label{tab:bucharest_rem_cond_sender}
	\end{mdframed}
\end{table}

Finally, for the target-conditioned model, the statistics that have a statistically significant impact on contact nomination likelihood are reciprocation ($p=0.0085$), exact reciprocation ($p<0.0001$), shared source ($p<0.0001$), the age of the nominator ($p=0.0391$), the discrepancy between nominator's and nominee's age ($p<0.0001$), the sex of the nominator ($p=0.0153$), the sex of the nominee ($p=0.0477$), the discrepancy between nominator's and nominee's sex ($p<0.0001$), the discrepancy between nominator's and nominee's job ($p<0.0001$), whether the nominator is employed in the medical field ($p<0.0001$), whether the nominee is employed in the medical field ($p=0.0036$), and the discrepancy between nominator's and nominee's medical employment status ($p=0.0005$). Reciprocation has a marginally negative effect on contact nomination likelihood (coefficient = -0.0294); exact reciprocation has a marginally positive effect (coefficient = 0.0202); shared source has a marginally positive effect (coefficient = 0.0163); nominator age and age discrepancy both have a marginally negative effect (coefficients = -0.0027 and -0.0237, respectively); nominator sex, nominee sex and sex discrepancy all have a slightly/moderately positive effect (coefficients = 0.1250, 0.1036 and 0.6887, respectively); job discrepancy has a moderately negative effect (coefficient = -0.4774); nominator medical employment and nominee medical employment both have a strongly/slightly negative effect, while medical employment status discrepancy has a slightly positive effect (coefficients = -1.5120, -0.1581 and 0.3031, respectively). Repetition, transitive tie and shared target are all-zero and thus weren't included in the Cox model.

\begin{table}[h]
	\centering
	\begin{mdframed}
		\begin{tabular}[width=\linewidth]{l|llll}
			\hline
			& \bfseries coef & \bfseries std err & $\mathbf{z}$ & $\mathbf{P>\lvert z \rvert}$\\
			\hline
			\csvreader[head to column names]{Tables/bucharest_rem_cond_receiver.csv}{}
			{\\ \csvcoliii & \csvcoliv & \csvcolv & \csvcolvi & \csvcolvii}\\
			\hline
		\end{tabular}
		\caption{Target-conditioned REM results for the Bucharest network. Repetition, transitive tie and shared target are all-zero and thus weren't included in the Cox model.}
		\label{tab:bucharest_rem_cond_receiver}
	\end{mdframed}
\end{table}

\paragraph{China network} With a 90\% confidence interval, the statistics that have a statistically significant impact on contact nomination likelihood are nomination activity ($p<0.0001$), reciprocation ($p<0.0001$), exact reciprocation ($p<0.0001$), cyclical tie ($p<0.0001$), shared source ($p<0.0001$), the age of the nominator ($p<0.0001$), the age of the nominee ($p<0.0001$), the sex of the nominator ($p<0.0001$), the sex of the nominee ($p<0.0001$), the discrepancy between nominator's and nominee's sex ($p<0.0001$), the place of residence of the nominator ($p<0.0001$), the place of residence of the nominee ($p<0.0001$), the discrepancy between nominator's and nominee's place of residence ($p<0.0001$), the place/occasion where the nominator might have been infected ($p<0.0001$), the place/occasion where the nominee might have been infected ($p<0.0001$), the discrepancy between nominator's and nominee's place/occasion ($p=0.0060$), the symptoms of the nominator ($p<0.0001$), the symptoms of the nominee ($p<0.0001$), the severity of the nominator's symptoms ($p<0.0001$), the severity of the nominee's symptoms ($p<0.0001$), the discrepancy in symptom severity ($p<0.0001$), the place where the nominator was registered as positive ($p<0.0001$), the place where the nominee was registered as positive ($p=0.0006$), and the discrepancy between nominator's and nominee's place of registration ($p<0.0001$). Nomination activity has a strongly negative effect on contact nomination likelihood (coefficient = -2.602); reciprocation has a slightly positive effect (coefficient = 0.2493); exact reciprocation has a marginally positive effect (coefficient = 0.0672); cyclical tie has a marginally negative effect (coefficient = -0.0642); shared source has a marginally positive effect (coefficient = 0.0150); nominator age and nominee age both have a marginally positive effect (coefficients = 0.0037 and 0.0053, respectively); nominator sex and nominee sex both have a slightly negative effect, while sex discrepancy has a slightly positive effect (coefficients = -0.1575, -0.2394 and 0.1519, respectively); nominator place of residence, nominee place of residence and discrepancy in place of residence all have a marginally/moderately negative effect (coefficients = -0.0010, -0.0009 and -0.4074, respectively); nominator and nominee place/occasion both have a marginally positive effect, while discrepancy in place/occasion has a marginally negative effect (coefficients = 0.0003, 0.0001 and -0.0576, respectively); nominator and nominee symptoms both have a marginally positive effect, while discrepancy in symptoms has a marginally negative effect (coefficients = 0.0026, 0.0016 and -0.0177, respectively); nominator symptom severity and nominee symptom severity both have a marginally/slightly positive effect, while discrepancy in symptom severity has a slightly negative effect (coefficients = 0.0601, 0.1692 and -0.2882, respectively); the nominator's place of registration, the nominee's place of registration and discrepancy in place of registration all have a marginally/very strongly negative effect (coefficients = -0.0010, -0.0009 and -5.928, respectively).



\section{Comparison of Relational Hyperevent Model Analysis across regions}
\label{sec:res_rhem}

\section{Discussion of results}
\label{sec:res_discussion}